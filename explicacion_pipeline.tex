\documentclass[12pt,a4paper]{article}
\usepackage[utf8]{inputenc}
\usepackage[spanish]{babel}
\usepackage{amsmath,amssymb,amsfonts}
\usepackage{geometry}
\usepackage{hyperref}
\usepackage{enumitem}
\usepackage{booktabs}
\usepackage{xcolor}
\usepackage{listings}
\usepackage{float}
\usepackage{tikz}
\usetikzlibrary{shapes,arrows.meta,positioning}

\geometry{margin=2.5cm}
\hypersetup{colorlinks=true,linkcolor=blue,citecolor=blue,urlcolor=blue}

\definecolor{critical}{RGB}{204,0,0}
\definecolor{warning}{RGB}{204,153,0}
\definecolor{info}{RGB}{0,102,204}

\lstset{
  basicstyle=\ttfamily\small,
  keywordstyle=\color{blue},
  commentstyle=\color{gray},
  breaklines=true,
  frame=single,
  language=Python
}

\title{\textbf{Documentaci\'on T\'ecnica Completa: Pipeline de Predicci\'on Financiera con Ensemble de Modelos} \\[0.5em]
\large An\'alisis Paso a Paso, Formulaci\'on Matem\'atica e Inconsistencias Detectadas}
\author{Generado autom\'aticamente}
\date{Febrero 2026}

\begin{document}
\maketitle
\tableofcontents
\newpage

% ============================================================
\section{Visi\'on General del Pipeline}
% ============================================================

Este proyecto implementa un pipeline de predicci\'on de precios financieros que:
\begin{enumerate}
    \item Descarga datos hist\'oricos de precios (acciones v\'ia \texttt{yfinance}, criptomonedas v\'ia \texttt{ccxt}).
    \item Calcula \textbf{log-returns} y los normaliza a $[0,1]$.
    \item Genera features: indicadores t\'ecnicos (TA-Lib) y macroecon\'omicos (FRED).
    \item Selecciona los top-$k$ features por \textbf{MIC} (Maximal Information Coefficient).
    \item Optimiza hiperpar\'ametros con \textbf{Optuna} usando \textbf{Walk-Forward Cross-Validation}.
    \item Entrena 4 modelos base: \textbf{LightGBM}, \textbf{CatBoost}, \textbf{TimeXer} (Transformer), \textbf{Moirai-MoE} (Foundation Model).
    \item Reconstruye predicciones a escala de log-return y de precio en USD.
    \item Genera un reporte HTML con gr\'aficas Plotly.
\end{enumerate}

\begin{figure}[H]
\centering
\begin{tikzpicture}[
    node distance=1.0cm,
    box/.style={rectangle, draw, rounded corners, fill=blue!10, minimum width=4cm, minimum height=0.8cm, align=center, font=\small},
    arrow/.style={-Stealth, thick}
]
\node[box] (data) {1. Descarga de Datos};
\node[box, below=of data] (lr) {2. Log Returns $+$ Normalizaci\'on};
\node[box, below=of lr] (feat) {3. Features: TA $+$ Macro};
\node[box, below=of feat] (mic) {4. Selecci\'on MIC top-$k$};
\node[box, below=of mic] (wf) {5. Walk-Forward CV};
\node[box, below=of wf] (train) {6. Optuna $+$ Entrenamiento};
\node[box, below=of train] (pred) {7. Predicci\'on $+$ Reconstrucci\'on};
\node[box, below=of pred] (report) {8. Reporte HTML};

\draw[arrow] (data) -- (lr);
\draw[arrow] (lr) -- (feat);
\draw[arrow] (feat) -- (mic);
\draw[arrow] (mic) -- (wf);
\draw[arrow] (wf) -- (train);
\draw[arrow] (train) -- (pred);
\draw[arrow] (pred) -- (report);
\end{tikzpicture}
\caption{Flujo general del pipeline.}
\end{figure}

\newpage
% ============================================================
\section{Paso 1: Descarga de Datos}
% ============================================================

\subsection{Acciones (yfinance)}
Se descargan datos OHLCV diarios mediante \texttt{yfinance} para los tickers:
\[
\mathcal{T}_{\text{yf}} = \{\texttt{KO}, \texttt{AAPL}, \texttt{NVDA}, \texttt{JNJ}, \texttt{\^{}GSPC}, \texttt{GC=F}, \texttt{CBOE}\}
\]
Per\'iodo: $[2020\text{-}01\text{-}01, \; 2025\text{-}12\text{-}31]$. Cada ticker se almacena como CSV en \texttt{data/tokens/}.

\subsection{Criptomonedas (CCXT/Binance)}
Se descargan datos OHLCV diarios v\'ia la API de Binance para:
\[
\mathcal{T}_{\text{cx}} = \{\texttt{BTC/USDT}, \texttt{ETH/USDT}\}
\]
Se usa paginaci\'on con l\'imite de 1000 velas por solicitud.

\subsection{Token Objetivo}
El token objetivo es \texttt{KO} (Coca-Cola). Se lee su CSV:
\[
\text{df} = \texttt{pd.read\_csv}(\texttt{KO\_2020-2025.csv})
\]

% ============================================================
\section{Paso 2: C\'alculo de Log-Returns y Normalizaci\'on}
% ============================================================

\subsection{Log-Returns}

Se define el log-return como:
\begin{equation}
\boxed{r_t = \ln\!\left(\frac{C_t}{C_{t+1}}\right)}
\label{eq:logret}
\end{equation}
donde $C_t$ es el precio de cierre en el instante $t$.

\textcolor{critical}{\textbf{INCONSISTENCIA CR\'ITICA \#1 (Secci\'on~\ref{sec:issues}):}} La convenci\'on est\'andar de log-return es:
\begin{equation}
r_t^{\text{est\'andar}} = \ln\!\left(\frac{C_t}{C_{t-1}}\right)
\label{eq:logret_std}
\end{equation}
En el c\'odigo se usa \texttt{shift(-1)} (shift hacia \emph{adelante}), lo que equivale a:
\[
r_t = \ln\!\left(\frac{C_t}{C_{t+1}}\right) = -\ln\!\left(\frac{C_{t+1}}{C_t}\right)
\]
Esto calcula el log-return \textbf{futuro con signo invertido}, introduciendo potencial \textbf{data leakage} ya que se usa informaci\'on del d\'ia $t+1$ para definir el target del d\'ia $t$.

\subsection{Normalizaci\'on Min-Max Global}

Se aplica normalizaci\'on Min-Max \textbf{global} (sobre toda la serie de log-returns):
\begin{equation}
\boxed{r_t^{\text{norm}} = \frac{r_t - r_{\min}}{r_{\max} - r_{\min}}}
\label{eq:minmax_global}
\end{equation}
donde $r_{\min} = \min_i r_i$ y $r_{\max} = \max_i r_i$ se calculan sobre \textbf{toda la serie} (train $+$ test).

\textcolor{critical}{\textbf{INCONSISTENCIA CR\'ITICA \#2:}} Esta normalizaci\'on se usa solo para visualizaci\'on pero puede confundir al lector. La normalizaci\'on que s\'i se aplica a los datos de entrenamiento/test se hace despu\'es con \texttt{MinMaxScaler} de scikit-learn de manera correcta (fit en train, transform en test).

% ============================================================
\section{Paso 3: Feature Engineering}
% ============================================================

\subsection{Indicadores T\'ecnicos (TA-Lib)}

Para cada rezago $r \in \{30, 31, \ldots, 364\}$ (335 rezagos), se calculan los siguientes indicadores sobre el precio de cierre $C$, precio alto $H$, precio bajo $L$ y volumen $V$:

\subsubsection{Indicadores de Tendencia}
\begin{itemize}
    \item \textbf{SMA} (Simple Moving Average):
    \begin{equation}
    \text{SMA}_r(t) = \frac{1}{r}\sum_{i=0}^{r-1} C_{t-i}
    \end{equation}
    
    \item \textbf{EMA} (Exponential Moving Average):
    \begin{equation}
    \text{EMA}_r(t) = \alpha \cdot C_t + (1-\alpha) \cdot \text{EMA}_r(t-1), \quad \alpha = \frac{2}{r+1}
    \end{equation}
    
    \item \textbf{TEMA} (Triple Exponential Moving Average):
    \begin{equation}
    \text{TEMA}_r = 3\,\text{EMA}_r - 3\,\text{EMA}(\text{EMA}_r) + \text{EMA}(\text{EMA}(\text{EMA}_r))
    \end{equation}
    
    \item \textbf{WMA} (Weighted Moving Average):
    \begin{equation}
    \text{WMA}_r(t) = \frac{\sum_{i=0}^{r-1}(r-i) \cdot C_{t-i}}{\sum_{i=0}^{r-1}(r-i)}
    \end{equation}
\end{itemize}

\subsubsection{Indicadores de Volatilidad}
\begin{itemize}
    \item \textbf{Bandas de Bollinger}:
    \begin{align}
    \text{middle}_r &= \text{SMA}_r \\
    \text{upper}_r &= \text{SMA}_r + 2\,\sigma_r \\
    \text{lower}_r &= \text{SMA}_r - 2\,\sigma_r
    \end{align}
    donde $\sigma_r$ es la desviaci\'on est\'andar m\'ovil de ventana $r$.
    
    \item \textbf{ATR} (Average True Range):
    \begin{equation}
    \text{TR}_t = \max\!\big(H_t - L_t,\; |H_t - C_{t-1}|,\; |L_t - C_{t-1}|\big)
    \end{equation}
    \begin{equation}
    \text{ATR}_r(t) = \frac{1}{r}\sum_{i=0}^{r-1}\text{TR}_{t-i}
    \end{equation}
\end{itemize}

\subsubsection{Indicadores de Momentum}
\begin{itemize}
    \item \textbf{MOM} (Momentum):
    \begin{equation}
    \text{MOM}_r(t) = C_t - C_{t-r}
    \end{equation}
    
    \item \textbf{RSI} (Relative Strength Index):
    \begin{equation}
    \text{RSI}_r = 100 - \frac{100}{1 + \frac{\overline{\Delta^+_r}}{\overline{\Delta^-_r}}}
    \end{equation}
    donde $\overline{\Delta^+_r}$ y $\overline{\Delta^-_r}$ son las medias de los cambios positivos y negativos en ventana $r$.
    
    \item \textbf{StochRSI}:
    \begin{equation}
    \text{StochRSI}_r = \frac{\text{RSI}_r - \min(\text{RSI}_r)}{\max(\text{RSI}_r) - \min(\text{RSI}_r)}
    \end{equation}
    Con par\'ametros $\text{fastk}=5$ y $\text{fastd}=3$.
    
    \item \textbf{KAMA} (Kaufman Adaptive Moving Average):
    Adapta su suavizado seg\'un la raz\'on se\~nal/ruido.
\end{itemize}

\subsubsection{Indicadores de Volumen}
\begin{itemize}
    \item \textbf{MFI} (Money Flow Index): RSI ponderado por volumen, ventana $r$.
    \item \textbf{AD} (Accumulation/Distribution):
    \begin{equation}
    \text{AD}_t = \text{AD}_{t-1} + \frac{(C_t - L_t) - (H_t - C_t)}{H_t - L_t} \cdot V_t
    \end{equation}
    \item \textbf{ADOSC} (A/D Oscillator): $\text{EMA}_3(\text{AD}) - \text{EMA}_{10}(\text{AD})$
    \item \textbf{OBV} (On-Balance Volume):
    \begin{equation}
    \text{OBV}_t = \text{OBV}_{t-1} + \text{sign}(C_t - C_{t-1}) \cdot V_t
    \end{equation}
\end{itemize}

\textbf{Total de features TA}: $335 \times 12 + 3 = 4{,}023$ columnas (antes de filtrar constantes).

\subsection{Features Macroecon\'omicos (FRED)}

Se obtienen las siguientes series de la Reserva Federal (FRED) y se alinean por fecha:

\begin{table}[H]
\centering
\begin{tabular}{ll}
\toprule
\textbf{C\'odigo FRED} & \textbf{Descripci\'on} \\
\midrule
\texttt{GDP} & Producto Interno Bruto \\
\texttt{CPIAUCSL} & \'Indice de Precios al Consumidor \\
\texttt{FEDFUNDS} & Tasa de Fondos Federales \\
\texttt{DGS10} & Tasa del Tesoro a 10 a\~nos \\
\texttt{SOFR} & Secured Overnight Financing Rate \\
\texttt{UNRATE} & Tasa de Desempleo \\
\bottomrule
\end{tabular}
\caption{Variables macroecon\'omicas.}
\end{table}

Se aplica \texttt{ffill} (forward fill) para rellenar valores faltantes (frecuencia mensual/trimestral $\rightarrow$ diaria).

% ============================================================
\section{Paso 4: Preparaci\'on de Datos y Split Train/Test}
% ============================================================

\subsection{Concatenaci\'on y Limpieza}
Se concatenan TA y Macro horizontalmente (por columnas). Se eliminan features constantes:
\[
\text{drop} = \{c \mid \max(c) - \min(c) < 10^{-8}\}
\]
Se reemplazan $\pm\infty$ por 0.

\subsection{Split Temporal 90/10}
\begin{equation}
\boxed{t_s = \lfloor 0.9 \cdot N \rfloor}
\end{equation}
\begin{align}
X_{\text{train}} &= X[0 : t_s], \quad y_{\text{train}} = r[0 : t_s] \\
X_{\text{test}} &= X[t_s : N], \quad y_{\text{test}} = r[t_s : N]
\end{align}

\subsection{Normalizaci\'on MinMax (Correcta)}
Se aplica \texttt{MinMaxScaler} de scikit-learn de forma correcta:

\paragraph{Para features $X$:}
\begin{equation}
x_{ij}^{\text{scaled}} = \frac{x_{ij} - \min_{\text{train}}(x_j)}{\max_{\text{train}}(x_j) - \min_{\text{train}}(x_j)}
\end{equation}
\texttt{fit} solo en $X_{\text{train}}$, luego \texttt{transform} en ambos conjuntos.

\paragraph{Para target $y$:}
\begin{equation}
y_i^{\text{scaled}} = \frac{y_i - \min_{\text{train}}(y)}{\max_{\text{train}}(y) - \min_{\text{train}}(y)}
\end{equation}
Mismo procedimiento: \texttt{fit} en $y_{\text{train}}$, \texttt{transform} en ambos.

% ============================================================
\section{Paso 5: Selecci\'on de Features por MIC}
% ============================================================

\subsection{Maximal Information Coefficient (MIC)}

El MIC captura dependencias tanto lineales como no lineales entre dos variables. Su c\'alculo:

\begin{enumerate}
    \item Se rankean $x$ e $y$:
    \begin{equation}
    \tilde{x}_i = \frac{\text{rank}(x_i) - 1}{n - 1}, \quad \tilde{y}_i = \frac{\text{rank}(y_i) - 1}{n - 1}
    \end{equation}
    donde $\tilde{x}, \tilde{y} \in [0, 1]$.
    
    \item Para cada par de particiones $(n_x, n_y)$ con $n_x \cdot n_y \leq B = n^{0.6}$:
    \begin{equation}
    I(X; Y) = \frac{1}{n}\left(\sum_{i,j} n_{ij}\log n_{ij} - \sum_i n_{i\cdot}\log n_{i\cdot} - \sum_j n_{\cdot j}\log n_{\cdot j}\right) + \log n
    \end{equation}
    
    \item Se normaliza la informaci\'on mutua:
    \begin{equation}
    M(n_x, n_y) = \frac{I(X; Y)}{\log\min(n_x, n_y)}
    \end{equation}
    
    \item El MIC es:
    \begin{equation}
    \boxed{\text{MIC}(X, Y) = \max_{n_x \cdot n_y \leq B} M(n_x, n_y)}
    \end{equation}
\end{enumerate}

Se seleccionan las top $k = 15$ features con mayor MIC respecto al target $y_{\text{train}}^{\text{scaled}}$.

% ============================================================
\section{Paso 6: Walk-Forward Cross-Validation}
% ============================================================

Se usa \texttt{SlidingWindowSplitter} de \texttt{sktime} con:
\begin{itemize}
    \item $k = 5$ folds
    \item Horizonte de validaci\'on: $\text{fh} = 30$ pasos
    \item Ventana de entrenamiento: $w = \lfloor 0.5 \cdot N_{\text{train}} \rfloor$
    \item Paso entre folds:
    \begin{equation}
    \text{step} = \max\!\left(1, \left\lfloor\frac{N_{\text{train}} - w - \text{fh}}{k - 1}\right\rfloor\right)
    \end{equation}
\end{itemize}

Cada fold $i$ genera:
\[
\text{Train}_i = [s_i, s_i + w), \quad \text{Val}_i = [s_i + w, s_i + w + \text{fh})
\]
donde $s_i = i \cdot \text{step}$.

% ============================================================
\section{Paso 7: Entrenamiento y Optimizaci\'on de Hiperpar\'ametros}
% ============================================================

Se usa \textbf{Optuna} (Tree-structured Parzen Estimator) para optimizar hiperpar\'ametros de cada modelo. El objetivo es minimizar el MAE medio sobre los folds de Walk-Forward:
\begin{equation}
\boxed{\theta^* = \arg\min_\theta \frac{1}{K}\sum_{i=1}^{K} \text{MAE}(y_{\text{val}}^{(i)}, \hat{y}_{\text{val}}^{(i)}(\theta))}
\end{equation}

\subsection{Modelo 1: LightGBM}

Gradient Boosting basado en \'arboles con histogramas. Hiperpar\'ametros optimizados:

\begin{table}[H]
\centering
\begin{tabular}{lll}
\toprule
\textbf{Par\'ametro} & \textbf{Rango} & \textbf{Tipo} \\
\midrule
\texttt{boosting\_type} & \{gbdt, dart, rf\} & Categ\'orico \\
\texttt{num\_leaves} & $[31, 200]$ & Entero \\
\texttt{max\_depth} & $[3, 12]$ & Entero \\
\texttt{learning\_rate} & $[0.01, 0.3]$ (log) & Flotante \\
\texttt{n\_estimators} & $[100, 1000]$ & Entero \\
\texttt{min\_child\_samples} & $[5, 100]$ & Entero \\
\texttt{subsample} & $[0.5, 1.0]$ & Flotante \\
\texttt{colsample\_bytree} & $[0.5, 1.0]$ & Flotante \\
\texttt{reg\_alpha} ($L_1$) & $[10^{-8}, 10]$ (log) & Flotante \\
\texttt{reg\_lambda} ($L_2$) & $[10^{-8}, 10]$ (log) & Flotante \\
\bottomrule
\end{tabular}
\end{table}

Funci\'on objetivo LightGBM:
\begin{equation}
\mathcal{L}_{\text{LGB}} = \sum_{i=1}^{N}\ell(y_i, \hat{y}_i) + \sum_{j=1}^{T}\Omega(f_j)
\end{equation}
donde $\Omega(f) = \gamma L + \frac{1}{2}\lambda \|w\|^2 + \alpha \|w\|_1$.

\subsection{Modelo 2: CatBoost}

Gradient Boosting con ``ordered boosting'' para reducir target leakage interno. Hiperpar\'ametros:

\begin{table}[H]
\centering
\begin{tabular}{lll}
\toprule
\textbf{Par\'ametro} & \textbf{Rango} & \textbf{Tipo} \\
\midrule
\texttt{iterations} & $[100, 1000]$ & Entero \\
\texttt{depth} & $[4, 10]$ & Entero \\
\texttt{learning\_rate} & $[0.01, 0.3]$ (log) & Flotante \\
\texttt{l2\_leaf\_reg} & $[10^{-3}, 10]$ (log) & Flotante \\
\texttt{min\_data\_in\_leaf} & $[5, 100]$ & Entero \\
\texttt{bootstrap\_type} & \{Bayesian, Bernoulli\} & Categ\'orico \\
\texttt{subsample} & $[0.5, 1.0]$ (si Bernoulli) & Flotante \\
\texttt{bagging\_temperature} & $[0.01, 10]$ (si Bayesian) & Flotante \\
\bottomrule
\end{tabular}
\end{table}

P\'erdida usada: RMSE.
\begin{equation}
\mathcal{L}_{\text{CB}} = \sqrt{\frac{1}{N}\sum_{i=1}^{N}(y_i - \hat{y}_i)^2}
\end{equation}

\subsection{Modelo 3: TimeXer (Transformer)}

TimeXer es una arquitectura Transformer adaptada para series temporales con variables ex\'ogenas. Componentes clave:

\subsubsection{Patch Embedding Endógeno}
La serie target se divide en patches de longitud $P$:
\begin{equation}
\text{N\'umero de patches} = \left\lfloor\frac{L_{\text{seq}}}{P}\right\rfloor
\end{equation}
Cada patch se proyecta linealmente a dimensi\'on $d_{\text{model}}$:
\begin{equation}
\mathbf{e}_i^{\text{en}} = W_{\text{patch}} \cdot \text{patch}_i + \text{PE}(i)
\end{equation}
donde PE es ``Positional Embedding'' sinusoidal:
\begin{align}
\text{PE}(pos, 2k) &= \sin\!\left(\frac{pos}{10000^{2k/d_{\text{model}}}}\right) \\
\text{PE}(pos, 2k+1) &= \cos\!\left(\frac{pos}{10000^{2k/d_{\text{model}}}}\right)
\end{align}
Se a\~nade un \textbf{global token} aprendible $\mathbf{g} \in \mathbb{R}^{d_{\text{model}}}$.

\subsubsection{Embedding Ex\'ogeno (Inverted)}
Las variables ex\'ogenas se tratan con embedding invertido:
\begin{equation}
\mathbf{e}^{\text{ex}} = W_{\text{inv}} \cdot X_{\text{exo}}^{\top} + b
\end{equation}
donde $X_{\text{exo}} \in \mathbb{R}^{L_{\text{seq}} \times (d-1)}$ y la proyecci\'on mapea la dimensi\'on temporal a $d_{\text{model}}$.

\subsubsection{Multi-Head Self-Attention}
\begin{equation}
\text{Attention}(Q, K, V) = \text{softmax}\!\left(\frac{QK^{\top}}{\sqrt{d_k}}\right)V
\end{equation}
\begin{equation}
\text{MultiHead}(Q, K, V) = \text{Concat}(\text{head}_1, \ldots, \text{head}_h)W^O
\end{equation}
donde $\text{head}_i = \text{Attention}(QW_i^Q, KW_i^K, VW_i^V)$.

\subsubsection{Cross-Attention (Global Token $\leftrightarrow$ Ex\'ogenas)}
El global token atiende a las variables ex\'ogenas mediante cross-attention:
\begin{equation}
\mathbf{g}' = \text{CrossAttn}\!\left(\mathbf{g}, \mathbf{e}^{\text{ex}}, \mathbf{e}^{\text{ex}}\right)
\end{equation}

\subsubsection{Feed-Forward}
\begin{equation}
\text{FFN}(x) = \text{Conv1d}_2\!\left(\sigma\!\left(\text{Conv1d}_1(x)\right)\right)
\end{equation}
donde $\sigma \in \{\text{ReLU}, \text{GELU}\}$, y las dimensiones son $d_{\text{model}} \to d_{\text{ff}} \to d_{\text{model}}$.

\subsubsection{Flatten Head}
\begin{equation}
\hat{y} = W_{\text{head}} \cdot \text{Flatten}(\text{enc\_out}) + b_{\text{head}}
\end{equation}
donde $\text{enc\_out} \in \mathbb{R}^{n_{\text{vars}} \times d_{\text{model}} \times (N_{\text{patches}}+1)}$.

\subsubsection{RevIN (Instance Normalization)}
Cuando \texttt{use\_norm=False} (como en este proyecto), \textbf{no se aplica}. Si estuviera activo:
\begin{equation}
\hat{x} = \frac{x - \mu_{\text{inst}}}{\sigma_{\text{inst}}}, \quad \hat{y}_{\text{final}} = \hat{y} \cdot \sigma_{\text{inst}} + \mu_{\text{inst}}
\end{equation}

\subsubsection{Hiperpar\'ametros TimeXer}

\begin{table}[H]
\centering
\begin{tabular}{lll}
\toprule
\textbf{Par\'ametro} & \textbf{Rango} & \textbf{Tipo} \\
\midrule
\texttt{patch\_len} & $\{4, 8, 12, 16, 24\}$ & Categ\'orico \\
\texttt{d\_model} & $\{64, 128, 256\}$ & Categ\'orico \\
\texttt{d\_ff} & $\{256, 512, 1024\}$ & Categ\'orico \\
\texttt{n\_heads} & $\{4, 8\}$ & Categ\'orico \\
\texttt{e\_layers} & $[1, 4]$ & Entero \\
\texttt{dropout} & $[0.0, 0.3]$ & Flotante \\
\texttt{factor} & $[1, 5]$ & Entero \\
\texttt{activation} & \{relu, gelu\} & Categ\'orico \\
\texttt{lr} & $[10^{-5}, 10^{-3}]$ (log) & Flotante \\
\texttt{batch\_size} & $\{16, 32, 64\}$ & Categ\'orico \\
\texttt{max\_epochs} & $[10, 50]$ & Entero \\
\texttt{patience} & $[5, 15]$ & Entero \\
\texttt{weight\_decay} & $[10^{-6}, 10^{-3}]$ (log) & Flotante \\
\bottomrule
\end{tabular}
\end{table}

P\'erdida: L1 (MAE).
\begin{equation}
\mathcal{L}_{\text{TX}} = \frac{1}{N}\sum_{i=1}^{N}|y_i - \hat{y}_i|
\end{equation}

Optimizador: AdamW con gradient clipping $\|\nabla\| \leq 1.0$.

\subsection{Modelo 4: Moirai-MoE (Foundation Model)}

Moirai-MoE es un \textbf{modelo fundacional pre-entrenado} (Salesforce) para series temporales, basado en Mixture of Experts. \textbf{No se re-entrena}; se usa zero-shot forecasting.

Funciona como:
\begin{enumerate}
    \item Se le pasa una ventana de contexto de longitud $L_{\text{ctx}}$ de la serie target (solo $y$, ignora $X$).
    \item Genera $N_{\text{samples}}$ trayectorias probabil\'isticas.
    \item La predicci\'on puntual es la \textbf{mediana} de las muestras:
    \begin{equation}
    \hat{y}_{t+1} = \text{median}\!\left(\{s^{(j)}_{t+1}\}_{j=1}^{N_{\text{samples}}}\right)
    \end{equation}
\end{enumerate}

Hiperpar\'ametros optimizados:

\begin{table}[H]
\centering
\begin{tabular}{lll}
\toprule
\textbf{Par\'ametro} & \textbf{Rango} & \textbf{Tipo} \\
\midrule
\texttt{context\_length} & $[64, \min(512, \text{max\_ctx})]$ (step 32) & Entero \\
\texttt{patch\_size} & $\{16, 32\}$ & Categ\'orico \\
\texttt{num\_samples} & $[50, 200]$ (step 25) & Entero \\
\texttt{batch\_size} & $\{16, 32, 64\}$ & Categ\'orico \\
\bottomrule
\end{tabular}
\end{table}

% ============================================================
\section{Paso 8: Predicci\'on en Test e Inversi\'on de Escala}
% ============================================================

\subsection{Predicci\'on en Escala Normalizada}

Cada modelo genera predicciones $\hat{y}_{\text{norm}} \in [0, 1]$ (escala normalizada MinMax). Estas predicciones se visualizan en la gr\'afica ``Predicciones -- Normalizado [0,1]''.

\subsection{Inversi\'on a Escala de Log-Return}

Se aplica la transformaci\'on inversa del \texttt{MinMaxScaler} del target:
\begin{equation}
\hat{r}_t = \hat{y}_t^{\text{norm}} \cdot (r_{\max}^{\text{train}} - r_{\min}^{\text{train}}) + r_{\min}^{\text{train}}
\end{equation}
donde $r_{\max}^{\text{train}}$ y $r_{\min}^{\text{train}}$ son los l\'imites del scaler ajustado en train.

\subsection{Reconstrucci\'on a Precio}

Se reconstruye el precio predicho a partir del log-return predicho:
\begin{equation}
\boxed{\hat{C}_t = C_{t-1} \cdot \exp(\hat{r}_t)}
\label{eq:recon}
\end{equation}
Donde $C_{t-1}$ es el precio de cierre real del d\'ia anterior. Esta funci\'on est\'a implementada con aceleraci\'on Numba cuando est\'a disponible.

% ============================================================
\section{Paso 9: M\'etricas de Evaluaci\'on}
% ============================================================

Se calculan las siguientes m\'etricas sobre la escala de \textbf{precio en USD}:

\begin{align}
\text{MSE} &= \frac{1}{N}\sum_{i=1}^{N}(C_i - \hat{C}_i)^2 \\
\text{RMSE} &= \sqrt{\text{MSE}} \\
\text{MAE} &= \frac{1}{N}\sum_{i=1}^{N}|C_i - \hat{C}_i| \\
\text{R}^2 &= 1 - \frac{\sum_{i=1}^{N}(C_i - \hat{C}_i)^2}{\sum_{i=1}^{N}(C_i - \bar{C})^2}
\end{align}

% ============================================================
\section{Paso 10: Generaci\'on de Reporte HTML}
% ============================================================

Se generan 14 gr\'aficas interactivas Plotly:

\begin{enumerate}
    \item Precio de cierre hist\'orico
    \item Log-returns crudos
    \item Retornos normalizados $[0,1]$
    \item Importancia de features (MIC) -- barras horizontales
    \item Walk-Forward folds (5 subgr\'aficas)
    \item Predicciones en escala normalizada
    \item Predicciones en escala log-return
    \item Predicciones en escala precio (USD)
    \item Serie completa $+$ predicciones overlay
    \item Zoom en zona de test
    \item--14. Un gr\'afico individual por modelo vs Real (USD)
    \item[15.] Comparaci\'on de m\'etricas (barras)
\end{enumerate}

% ============================================================
\section{An\'alisis de Inconsistencias Detectadas}
\label{sec:issues}
% ============================================================

\subsection*{Resumen R\'apido de Severidad}

\begin{table}[H]
\centering
\begin{tabular}{clp{8cm}}
\toprule
\textbf{\#} & \textbf{Severidad} & \textbf{Descripci\'on} \\
\midrule
1 & \textcolor{critical}{\textbf{CR\'ITICA}} & Log-return calculado con \texttt{shift(-1)} (data leakage + signo invertido) \\
2 & \textcolor{warning}{\textbf{MODERADA}} & Moirai-MoE usa $y_{\text{test}}$ real en su contexto de ventana (leakage sutil) \\
3 & \textcolor{warning}{\textbf{MODERADA}} & TimeXer usa $y_{\text{test}}$ real en \texttt{full\_data} para ventana de predicci\'on (leakage) \\
4 & \textcolor{info}{\textbf{MENOR}} & El modelo predice sobre log-returns normalizados, no sobre log-returns crudos \\
5 & \textcolor{info}{\textbf{MENOR}} & No hay LSTM en el c\'odigo (a pesar de la pregunta del usuario) \\
6 & \textcolor{info}{\textbf{MENOR}} & Ruta absoluta hardcodeada para lectura de datos \\
\bottomrule
\end{tabular}
\end{table}

% ---- Inconsistencia 1 ----
\subsection{Inconsistencia \#1 -- Log-Return con \texttt{shift(-1)} {\normalfont\textcolor{critical}{[CR\'ITICA]}}}

\textbf{C\'odigo (l\'inea 29 de \texttt{main2.py}):}
\begin{lstlisting}
lc = np.log(df['Close'] / df['Close'].shift(-1)).dropna()
\end{lstlisting}

\textbf{Problema:} \texttt{shift(-1)} desplaza la serie \textbf{hacia adelante}, es decir:
\[
r_t = \ln\!\left(\frac{C_t}{C_{t+1}}\right)
\]
Esto tiene dos problemas:
\begin{enumerate}[label=(\alph*)]
    \item \textbf{Data leakage:} El valor del target $r_t$ en el instante $t$ depende de $C_{t+1}$, que es informaci\'on futura. Cuando el modelo aprende a predecir $r_t$ a partir de features del d\'ia $t$ (o anteriores), est\'a aprendiendo a predecir algo que incluye informaci\'on del futuro en su definici\'on.
    
    \item \textbf{Signo invertido:} El log-return resultante tiene signo opuesto al convencional. Si el precio sube ($C_{t+1} > C_t$), el log-return ser\'a \textbf{negativo}, lo cual es contraintuitivo.
\end{enumerate}

\textbf{Correcci\'on:}
\begin{lstlisting}
lc = np.log(df['Close'] / df['Close'].shift(1)).dropna()
\end{lstlisting}
Esto da $r_t = \ln(C_t / C_{t-1})$, que es la definici\'on est\'andar.

\textcolor{critical}{\textbf{Nota:}} Sin embargo, como todo el pipeline (entrenamiento, predicci\'on, reconstrucci\'on) opera bajo la misma definici\'on invertida, la \textbf{reconstrucci\'on de precio funciona correctamente de forma interna}. La f\'ormula $\hat{C}_t = C_{t-1} \cdot \exp(\hat{r}_t)$ sigue siendo v\'alida porque el modelo aprende la relaci\'on consistente (aunque invertida). El impacto real de esta inconsistencia es principalmente \textbf{sem\'antico} y en la \textbf{interpretabilidad}, no necesariamente en la precisi\'on del modelo.

% ---- Inconsistencia 2 ----
\subsection{Inconsistencia \#2 -- Data Leakage en Moirai-MoE {\normalfont\textcolor{warning}{[MODERADA]}}}

\textbf{C\'odigo (\texttt{moraiMOE\_model.py}, l\'ineas 115, 136--144):}
\begin{lstlisting}
full_series = np.concatenate([y_train_arr, y_test_arr])
# ...
for i in range(len(y_test_arr)):
    ctx_data = full_series[ctx_start:ctx_end]
\end{lstlisting}

\textbf{Problema:} Durante la predicci\'on punto a punto en test, la ventana de contexto se extrae de \texttt{full\_series} que contiene \textbf{tanto train como test}. Al predecir $\hat{y}_{t}$, la ventana $[\text{ctx\_start}, t)$ puede incluir valores \textbf{reales} de $y$ ya observados en el test set (es decir, $y_{t-1}, y_{t-2}, \ldots$ del test set, no predicciones previas).

Esto introduce un form de \textbf{leakage temporal suave}: el modelo tiene acceso a valores reales que en producci\'on no estar\'ian disponibles (se tendr\'ian las predicciones propias, no los valores reales).

\textbf{Impacto:} Las m\'etricas del modelo Moirai son \textbf{optimistas} comparadas con un escenario real de despliegue.

% ---- Inconsistencia 3 ----
\subsection{Inconsistencia \#3 -- Data Leakage en TimeXer {\normalfont\textcolor{warning}{[MODERADA]}}}

\textbf{C\'odigo (\texttt{timexer\_model.py}, l\'ineas 250--255, 341--350):}
\begin{lstlisting}
full_data = np.concatenate([train_data, test_data], axis=0)
# ...
for i in range(len(test_data)):
    x_window = full_data[window_start:window_end]
\end{lstlisting}

\textbf{Problema id\'entico al \#2:} La ventana de entrada al Transformer durante test incluye filas de \texttt{test\_data} con los \textbf{valores reales de $y$}. La \'ultima columna de \texttt{full\_data} es $y$ (target), por lo que al crear la secuencia $[t - L_{\text{seq}}, t)$, las filas que caen en el rango de test contienen $y_{\text{test}}$ real en la \'ultima columna.

\textbf{Impacto:} Las m\'etricas de TimeXer tambi\'en son \textbf{optimistas}.

% ---- Inconsistencia 4 ----
\subsection{Inconsistencia \#4 -- Dominio de Predicci\'on {\normalfont\textcolor{info}{[MENOR]}}}

\textbf{Los modelos predicen log-returns normalizados $\hat{y} \in [0,1]$} (escala MinMax), no log-returns crudos directamente. Esto es v\'alido y no es un error per se, pero es importante entender que:

\begin{enumerate}
    \item La inversi\'on de escala se aplica correctamente con \texttt{sct.inverse\_transform()}.
    \item Las m\'etricas finales se calculan en escala de precio USD, lo cual es correcto.
    \item Los log-returns crudos se recuperan correctamente antes de la reconstrucci\'on de precio.
\end{enumerate}

Esta observaci\'on es solo clarificativa; el pipeline es consistente en este aspecto.

% ---- Inconsistencia 5 ----
\subsection{Inconsistencia \#5 -- No Existe LSTM {\normalfont\textcolor{info}{[MENOR]}}}

El usuario pregunt\'o sobre si ``la LSTM est\'a mal''. \textbf{No hay ning\'un modelo LSTM en el c\'odigo.} Los cuatro modelos son:

\begin{enumerate}
    \item LightGBM -- Gradient Boosting (\'arboles)
    \item CatBoost -- Gradient Boosting (\'arboles)
    \item TimeXer -- Transformer (atenci\'on, NO recurrencia)
    \item Moirai-MoE -- Foundation Model pre-entrenado (NO LSTM)
\end{enumerate}

\textbf{TimeXer} usa multi-head self-attention, no celdas LSTM.

% ---- Inconsistencia 6 ----
\subsection{Inconsistencia \#6 -- Ruta Absoluta Hardcodeada {\normalfont\textcolor{info}{[MENOR]}}}

\textbf{L\'inea 29:}
\begin{lstlisting}
df = pd.read_csv(rf"C:\Users\hibra\Desktop\TT\data\tokens\{TOKEN}_2020-2025.csv")
\end{lstlisting}
La ruta apunta a \texttt{Desktop\textbackslash TT} pero el proyecto est\'a en \texttt{TT2\textbackslash TT}. Esto podr\'ia causar que se lean datos de una versi\'on diferente del proyecto.

% ============================================================
\section{Resumen de Correctitud del Pipeline}
% ============================================================

\begin{table}[H]
\centering
\begin{tabular}{lcc}
\toprule
\textbf{Componente} & \textbf{Correcto?} & \textbf{Nota} \\
\midrule
Descarga de datos & \checkmark & Funcional \\
Log-return (direcci\'on) & $\times$ & \texttt{shift(-1)} en vez de \texttt{shift(1)} \\
Normalizaci\'on visual & \checkmark & Solo para gr\'aficas \\
Feature engineering TA & \checkmark & 4,000+ features \\
Features macro & \checkmark & FRED + ffill \\
MinMaxScaler train/test & \checkmark & fit en train, transform en test \\
MIC feature selection & \checkmark & Solo datos train \\
Walk-Forward CV & \checkmark & Respeta temporalidad \\
Optuna optimizaci\'on & \checkmark & Minimiza MAE \\
LightGBM & \checkmark & Sin leakage interno \\
CatBoost & \checkmark & Sin leakage interno \\
TimeXer predicci\'on test & $\times$ & Leakage: $y_{\text{test}}$ real en ventana \\
Moirai-MoE predicci\'on test & $\times$ & Leakage: $y_{\text{test}}$ real en contexto \\
Inversi\'on de escala & \checkmark & Correcto \\
Reconstrucci\'on a precio & \checkmark & Internamente consistente \\
M\'etricas & \checkmark & MSE, RMSE, MAE, R$^2$ \\
Reporte HTML & \checkmark & 14+ gr\'aficas Plotly \\
\bottomrule
\end{tabular}
\end{table}

\end{document}
